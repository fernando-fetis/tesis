\chapter*{Resumen}

En los últimos años, los modelos de difusión han emergido como una poderosa clase de modelos generativos, alcanzando el estado del arte en diversos dominios, particularmente en la generación de imágenes. Estos modelos operan corrompiendo gradualmente una muestra proveniente de una distribución desconocida $\ptrue$ hasta llegar a otra distribución $\pprior$ de la cual es fácil generar muestras. Para lograr generar nuevas muestras a partir de la distribución $\ptrue$, se entrena una red neuronal que aprenda a revertir este proceso. De esta forma, se puede generar una muestra desde $\ptrue$ comenzando desde una muestra de $\pprior$ y aplicando el proceso reverso aprendido hasta llegar a una muestra de $\ptrue$.

A pesar de su impresionante rendimiento, los modelos de difusión enfrentan ciertas limitaciones como, por ejemplo, no permitir transformar la distribución inicial $\ptrue$ en otra distribución arbitraria o no admitir un entrenamiento con datos no emparejados. Para abordar estas limitaciones, se han propuesto enfoques alternativos, entre los cuales destaca el problema del puente de Schrödinger. Dadas dos distribuciones de probabilidad $\mu,\nu\in\probmeasure{\xspace}$ y un proceso estocástico de referencia $R=(R_t)_{t\in[0,1]}$, el problema del puente de Schrödinger consiste en encontrar el proceso estocástico $P=(P_t)_{t\in[0,1]}$ que más se asemeje a $R$ (en el sentido de la entropía relativa) y cuyas marginales en los tiempos $t=0$ y $t=1$ sean $\mu$ y $\nu$ respectivamente.

Si bien este problema soluciona las limitaciones antes mencionadas, la literatura usual a menudo presenta el problema del puente de Schrödinger utilizando definiciones y aspectos técnicos de la teoría matemática subyacente, los cuales muchas veces son prescindibles para el enfoque práctico que se sigue en el modelamiento generativo. Estos aspectos suelen complicar el entendimiento del problema, creando una barrera de entrada para muchos investigadores en la comunidad de aprendizaje automático. En consecuencia, el objetivo de esta tesis es cerrar esta brecha proporcionando un tratamiento exhaustivo y accesible del tema, estableciendo una transición natural desde los modelos de difusión hasta el problema del puente de Schrödinger, donde el punto de conexión será la teoría del transporte óptimo, la cual resultará ser, en su versión regularizada, equivalente al problema del puente de Schrödinger.

Esta tesis está organizada en tres capítulos. En el \autoref{dm} se realizará una exploración profunda de los modelos de difusión, sentando las bases para comprender sus fortalezas y limitaciones. Luego, en el \autoref{ot} se introducirá el problema de transporte óptimo, el cual consiste en transformar una distribución de probabilidad en otra, minimizando un cierto funcional de costo. Este capítulo introduce conceptos clave como la distancia de Wasserstein y sus propiedades geométricas, mostrando la robustez de este problema de transporte en el espacio de las medidas de probabilidad.

Finalmente, en el \autoref{eot_sbp} se estudiará una versión regularizada del problema de transporte, la cual resultará ser equivalente a al problema del puente de Schrödinger (en su versión estática), permitiendo heredar toda la teoría del transporte óptimo a este problema.

Por otro lado, el apéndice contiene algunas definiciones resultados utilizados a lo largo del documento. Además, el marco de trabajo en este documento será siempre $\R^d$ o un conjunto finito, aunque las definiciones y demostraciones son fácilmente generalizables a marcos más generales como espacios polacos. Con respecto a las demostraciones realizadas, estas buscan ser instructivas más que rigurosas, por lo que muchos resultados serán demostrados en su versión discreta y luego generalizados, sin demostración, al caso continuo. Todos los archivos de esta tesis (manuscrito, simulaciones y otros) se encuentran en el repositorio \href{https://github.com/fernando-fetis/tesis}{\texttt{https://github.com/fernando-fetis/tesis}}.