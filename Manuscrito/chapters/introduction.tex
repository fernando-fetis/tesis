\chapter*{Introducción}

En la última década, los \textit{modelos generativos} han cobrado gran relevancia en el campo de la inteligencia artificial, destacándose por su capacidad para modelar datos complejos y sintetizar ejemplos realistas en diversas aplicaciones, desde la generación de imágenes hasta la síntesis de audio. Dentro de esta área, los \textit{modelos de difusión} han surgido como una familia prometedora debido a su robustez y a la alta calidad en la generación de datos. Estos modelos funcionan distorsionando progresivamente datos provenientes de una distribución inicial hasta alcanzar una distribución de ruido, y luego aprenden a revertir este proceso para generar nuevas muestras.

Sin embargo, los modelos de difusión presentan ciertas limitaciones significativas, tales como el alto costo computacional asociado a las simulaciones iterativas y la dificultad para ajustar adecuadamente las transiciones reversas. Estas limitaciones han motivado la exploración de enfoques alternativos. En este trabajo se estudia el \textit{problema del puente de Schrödinger}, una formulación basada en la teoría del transporte óptimo entrópico, que ha ganado popularidad debido a su capacidad generativa y a su solidez teórica. El puente de Schrödinger permite formular un problema de interpolación estocástica entre distribuciones, ofreciendo una perspectiva distinta y, en muchos casos, más eficiente para abordar problemas generativos complejos.

\section*{Contribuciones}

Las principales contribuciones de esta tesis son las siguientes:
\begin{enumerate}
    \item Desarrollo exhaustivo de los aspectos teóricos y prácticos de los modelos de difusión, el transporte óptimo y el problema del puente de Schrödinger. Se presenta una integración coherente de estos conceptos, conectando los temas de forma natural y facilitando una comprensión unificada.
    \item Formulación detallada del transporte óptimo para justificar de manera sólida el uso de los puentes de Schrödinger como una extensión natural. Esta formulación permite una transición fluida desde los modelos de difusión hacia los métodos de interpolación estocástica.
    \item Revisión crítica de la literatura reciente, incorporando resultados y técnicas modernas tanto en el desarrollo teórico como en las implementaciones numéricas de los modelos generativos.
    \item Desarrollo de un conjunto extenso de simulaciones numéricas y arquitecturas neuronales, incluyendo:
          \begin{itemize}
              \item \textbf{Modelos generativos tradicionales}: se implementó un modelo de autoencoder variacional y una red generativa adversarial para introducir los temas de generación neuronal y modelos de variable latente.
              \item \textbf{Modelos de difusión}: en el estudio de este tipo de modelos, se implementaron los modelos denoising diffusion probabilistic models (DDPM), score matching (DSM), dinámica de Langevin y técnicas de guidance. Además, se implementaron las arquitecturas neuronales U-Net y DiT, las cuales constituyen el estado del arte en redes neuronales para modelos de difusión.
              \item \textbf{Transporte óptimo y puentes de Schrödinger}: para el estudio de este problema, se simuló una solución del problema de Kantorovich (tanto en el caso discreto como en el continuo) y se realizó una simulación de la formulación de Benamou-Brenier. Además, se implementó una interpolación de McCann, el algoritmo de Sinkhorn y una red WGAN.
          \end{itemize}
    \item Implementación de un marco teórico y experimental autocontenido, que permite una fácil extensión a los últimos avances en métodos para resolver el problema del puente de Schrödinger.
\end{enumerate}

Es importante destacar que uno de los objetivos principales de esta tesis es proporcionar un estudio autocontenido del problema del puente de Schrödinger, el cual presenta múltiples formulaciones equivalentes, muchas de ellas utilizando resultados técnicos avanzados. Para facilitar la comprensión y el desarrollo de estas formulaciones, se ha incluido un apéndice que cubre de manera detallada los resultados necesarios de cálculo estocástico y teoría de la medida, proporcionando un soporte técnico robusto para los desarrollos teóricos presentados.